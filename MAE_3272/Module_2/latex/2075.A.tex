\documentclass[11pt]{article}
 \usepackage[margin=1in]{geometry} 
\usepackage{amsmath,amsthm,amssymb,amsfonts}
 
 

 
\begin{document}
 

 
\title{MAE 3272 Module 2 Abstract}
\author{Samuel Tome - 2075.A \\ Obtained reviews from 2075.R1 and 2075.R2}
\maketitle

% This one goes into the hand-in box

% Describe succinctly what you did in this Module. It should be written in sentence form. It should not exceed 300 words. It should be devoid of figures or graphs. It should succinctly describe your load cell design, the calibration procedure, your calibration result and its comparison to what you predicted. Finally, it should also describe how well your S-gage load sensor performed in determining an unknown weight.

Force transducers are an invaluable engineering tool for their ability to directly measure physical forces quickly and precisely.  This lab module sought to design, construct, and calibrate an S-beam load cell using conventional strain gauges arranged in a Wheatstone full bridge configuration with LabView data acquisition software.  The supplied beams of the S-beam load cell were $4$" long by $^1/_2$" wide by $^1/_2$" high.  To maximize sensitivity of the load cell, a shear and bending moment analysis of the system determined the optimal strain gauge placement to be on the tops and bottoms of the ends of each beam because this is where the strain is the largest.  In order to satisfy the full bridge circuit, the positions were also chosen such that two gauges would experience tension while the other two experienced compression.  The shear and bending moment analysis, along with the beam axial stress equation and material performance data, also showed that the maximum load the cell could sustain without yielding was 385 [lbf].  To calibrate the cell, a "load vs. output voltage" curve was generated by placing a number of successive known weights onto the cell and recording the resultant voltage change through the Wheatstone bridge.  A linear least squares fit was applied to these data, from which an unknown load was determined using the average of its characteristic output voltages when examined using the load cell.  The experimental least squares line had a slope of $-.011348 $ [mV/lbf] and an intercept of $-0.56323$ [volts], whereas the computed sensitivity from beam bending, gauge factor, and input voltage was $ -.0294 $ [mV/lbf].  One possible reason for this factor of three discrepancy could be how close the gauges were to the spacer cubes.  Because of how close the gauges are to what can effectively be considered fixed supports, the traditional long beam bending equations don't apply to this region.  Despite this, the linearity of the "output voltage vs. load" curve was still excellent, with a third standard deviation of $0.057$ [volts] or $0.502$ [lbf].  This was confirmed during testing of a 14.4 [lbf] "unknown load", which was repeated five times with a bias of only $0.053$ [lbf], yielding an average measurement of $14.38$ [lbf].  The excellent repeatability, linear fit, and prediction of unknown loads thus confirm the efficacy of the S-beam load cell as a force transducer.


\end{document}