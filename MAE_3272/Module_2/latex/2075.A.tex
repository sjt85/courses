\documentclass[10pt]{article}
 \usepackage[margin=1in]{geometry} 
\usepackage{amsmath,amsthm,amssymb,amsfonts}
 
 

 
\begin{document}
 

 
\title{MAE 3272 Module 2 Abstract}
\author{2075.A}
\date{}
\maketitle

% Describe succinctly what you did in this Module. It should be written in sentence form. It should not exceed 300 words. It should be devoid of figures or graphs. It should succinctly describe your load cell design, the calibration procedure, your calibration result and its comparison to what you predicted. Finally, it should also describe how well your S-gage load sensor performed in determining an unknown weight.

This lab module sought to design, construct, and calibrate an S-beam load cell using conventional strain gauges arranged in a Wheatstone full bridge configuration with LabView data acquisition software.  The supplied beams of the S-beam load cell were $4$" long by $^1/_2$" wide by $^1/_2$" high.  A shear and bending moment analysis of the system showed that the largest stresses and strains would occur on the tops and bottoms of the ends of each beam.  Therefore to maximize sensitivity the four gauges were positioned directly inside of the spacers on the top and bottom of beams, with the centers of the gauges being $1 \: ^3/_{16}$" from the middle of the beams.  In order to satisfy the full bridge circuit, the positions were also chosen such that two gauges would experience tension while the other two experienced compression.  The shear and bending moment analysis, along with the beam axial stress equation and material performance data, also showed that the maximum load the cell could sustain without yielding was 385 [lbf].  To calibrate the cell, a number of successive known weights were loaded onto the cell and the resultant voltage through the Wheatstone bridge read using the LabView data acquisition instruments.  A linear least squares fit was applied to the "load vs. output voltage" data, from which an unknown load could thus be determined using its characteristic output voltage.  The experimental least squares line had a slope of $-.011348 $ [mV/lbf] and an intercept of $-0.56323$ [volts], whereas the computed sensitivity from beam bending, gauge factor, and input voltage was $ -.0294 $ [mV/lbf].  One possible reason for this factor of 3 discrepancy could be how close the gauges are to the spacer cubes.  When the bolts holding the load cell together are very tight, this forces the ends of the beams to not deflect as much, with a zero slope end condition imposed.  Consequently, the traditional beam stress equation doesn't apply to regions very close to rigid supports such as the bolts and spacers.  Despite this, the linearity of the "output voltage vs. load" curve was still excellent, with a third standard deviation of $0.057$ [volts] or $0.502$ [lbf].  This was confirmed during testing of a 14.4 [lbf] "unknown load", which was repeated five times with a bias of only $0.053$ [lbf], yielding an average measurement of $14.38$ [lbf].  The excellent repeatability, linear fit, and prediction of unknown loads thus confirm the efficacy of the S-beam load cell as a force transducer.


\end{document}