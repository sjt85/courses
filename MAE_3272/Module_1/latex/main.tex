\documentclass[10pt]{article}
 \usepackage[margin=1in]{geometry} 
\usepackage{amsmath,amsthm,amssymb,amsfonts}
 
\newcommand{\N}{\mathbb{N}}
\newcommand{\Z}{\mathbb{Z}}
 
\newenvironment{problem}[2][Problem]{\begin{trivlist}
\item[\hskip \labelsep {\bfseries #1}\hskip \labelsep {\bfseries #2.}]}{\end{trivlist}}
%If you want to title your bold things something different just make another thing exactly like this but replace "problem" with the name of the thing you want, like theorem or lemma or whatever
 
\begin{document}
 
%\renewcommand{\qedsymbol}{\filledbox}
%Good resources for looking up how to do stuff:
%Binary operators: http://www.access2science.com/latex/Binary.html
%General help: http://en.wikibooks.org/wiki/LaTeX/Mathematics
%Or just google stuff
 
\title{MAE 3272 Module 1 Abstract}
\author{Samuel Tome - 1008.A \\ Obtained review from 1020.R1}
\maketitle

% ALSO: On this copy of the Abstract, at the top of the page: Please put your name, your assigned, special random number and the Peer Reviews you obtained (use simply R1 and/or R2) to let the grader know which peer review(s) you obtained. Many thanks. . . . WS.

This lab exercise sought to investigate the mechanical properties of Al 6061-T0 and Al 6061-T6 aluminum and AISI 1018 cold rolled and annealed steel.  Hardness, elastic moduli, yield and ultimate strengths, reduction of area, and elongation at fracture were determined using ASTM standards for hardness indentation, ultrasonic, and tensile testing.  The results were then compared with published industrial values.  The ultrasonic tests measured round trip travel time of both longitudinal and shear waves using V109 and V154 transducers, respectively.   The modulus of elasticity, bulk modulus, and Poisson's ratio of the four materials were then determined from the wavelengths obtained via oscilloscope and the specimens' measured thicknesses and densities.    Tensile tests were performed next on test coupons of each material.  A load frame and load cell along with an extensometer were used to obtain the load vs. displacement curves for the four coupons.  These curves were translated into stress-strain curves from which the Young's moduli, yield, and ultimate strengths were obtained.  The reduction in area and elongation at fracture were determined using calipers and a ruler.  Lastly, hardness testing was performed on one inch rounds of each material on an automatic testing machine, using a combination of pre-loads and indenters that would allow the hardnesses to fall into appropriate ranges.  At least ten runs were performed on each specimen, which were then averaged together and the variances between the runs found.  The experimentally-derived ultrasound data matched well with published industrial values, producing results less than 10\% off from industrially-listed values for all materials but the 6061-T6 aluminum, which likely had issues in setting the horizontal and vertical scales used on the oscilloscope.  The results for the tension tests deviated more significantly from published values than the ultrasound results did - the tension test results were up to 15\% or more off from the published values.  This discrepancy is largely due to the manual process by which the slopes of the stress-strain curves were fitted to find the Young's moduli and the subsequent 0.2\% offset method used to determine yield strengths.  Results from hardness testing, though with a fairly large standard deviation between different runs, overall agreed excellently with published values for the two processed specimens, with both being only a couple percentage points off from published values once the runs are averaged together.  The annealed steel, however, varied over 20\% from its published hardness value.  This could be due to subtle differences in the annealing techniques used between the specimens tested in this lab and those used for the industrial tests.  The 6061-T0 aluminum was too soft to test for hardness.  The results of these experiments may facilitate the conclusion that adding processing techniques such as heat treating or rolling to one of the alloys examined significantly increases its performance in hardness and strength while affecting its density negligibly.  For instance, in the case of 6061 aluminum the measured yield and ultimate strengths increase from 89 MPa and 99 MPa, respectively, to 251 MPa and 299 MPa with a T6 heat treatment.  The measured yield and ultimate strengths of the 1018 steel increase from 370 MPa and 440 MPa to 548 MPa and 554 MPa with cold rolling, and the hardness increases from 100 to 176 on a Vickers scale.  For this reason, the processing techniques examined could make these processed materials more appealing material candidates in many situations.


\end{document}